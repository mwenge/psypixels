\begin{figure}[H]
    \centering
    \foreach \l in {0,...,10}
    {
      \includegraphics[width=4.2cm]{linewidth/pattern\l-45.png}%
    }%
    \caption{
      Line Mode with Pulse Width at Maximum
      }
\end{figure}
\clearpage
\section*{line mode} 
\label{sec:linemode}
\lstset{style=6502Style}

\begin{definition}[Jeffrey Says]
\setlength{\intextsep}{0pt}%
\setlength{\columnsep}{3pt}%
\begin{wrapfigure}{l}{0.12\textwidth}
\includegraphics[width=\linewidth]{src/callout/psych.png} 
\end{wrapfigure}
\textbf{Press L to turn on and off} the Line Mode - a bit like drawing with the Aurora Borealis.\\
\textbf{Press W to adjust line width:} Sets the width of the lines produced in Line Mode.
\\
\\
\end{definition}
'Line Mode' is a completely different mode of painting. 

\clearpage
\textbf{Lines 1189-1231. \icode{\textbf{JustLPressed}}} 
\begin{lstlisting}[caption=From \icode{CheckKeyboardInput}.]
JustLPressed   
        ; 'L' pressed. Turn line mode on or off.
        LDA lineModeActivated
        EOR #$01
        STA lineModeActivated
        ASL 
        ASL 
        ASL 
        ASL 
        TAY 

        ; Briefly display the new linemode on the bottom of the screen.
        JSR ClearLastLineOfScreen
        LDX #$00
_Loop   LDA lineModeSettingDescriptions,Y
        STA lastLineBufferPtr,X
        INY 
        INX 
        CPX #$10
        BNE _Loop
        JMP WriteLastLineBufferToScreen
        ; Returns
\end{lstlisting}

\clearpage

\textbf{Lines 1189-1231. \icode{\textbf{JustLPressed}}:} 
Pressing \icode{L} toggles line mode on or off. Line mode's on/off state is 
stored in \icode{lineModeActivated}, with \icode{\$00} meaning 'off' and
\icode{\$01} meaning 'on'. This means we can use a neat, economical trick
for toggling the value every time the user presses the 'L' key:

\begin{lstlisting}
        LDA lineModeActivated
        EOR #$01
        STA lineModeActivated
\end{lstlisting}

The \icode{EOR} statement performs an exclusive-or that has the neat property of
turning \icode{\$00} into \icode{\$01} and \icode{\$01} into \icode{\$00} - equivelent
to switching a value on or off.

This is what the the bit by bit operation looks like when '\icode{EOR \$01}' turns
\icode{trackingActivated} 'on':

\begin{figure}[H]
  {
    \setlength{\tabcolsep}{3.0pt}
    \setlength\cmidrulewidth{\heavyrulewidth} % Make cmidrule = 
    \begin{adjustbox}{width=7cm,center}

      \begin{tabular}{rllllllll}
        \toprule
        Byte & Bit 7 & Bit 6 & Bit 5 & Bit 4 & Bit 3 & Bit 2 & Bit 1 & Bit 0        \\
        \midrule
        \$00 & 0 & 0 & 0 & 0 & 0 & 0 & 0 & 0 \\
        \$01 & 0 & 0 & 0 & 0 & 0 & 0 & 0 & 1 \\
        \midrule
        Result & 0 & 0 & 0 & 0 & 0 & 0 & 0 & 1 \\
        \addlinespace
        \bottomrule
      \end{tabular}

    \end{adjustbox}

  }\caption*{X-OR'ing \$01 and \$00 gives \$01, the 'on' value for \icode{lineModeActivated}.}
\end{figure}

And this is what it looks like when \icode{EOR \$01} turns \icode{lineModeActivated} 'off' again:
\begin{figure}[H]
  {
    \setlength{\tabcolsep}{3.0pt}
    \setlength\cmidrulewidth{\heavyrulewidth} % Make cmidrule = 
    \begin{adjustbox}{width=7cm,center}

      \begin{tabular}{rllllllll}
        \toprule
        Byte & Bit 7 & Bit 6 & Bit 5 & Bit 4 & Bit 3 & Bit 2 & Bit 1 & Bit 0        \\
        \midrule
        \$01 & 0 & 0 & 0 & 0 & 0 & 0 & 0 & 1 \\
        \$01 & 0 & 0 & 0 & 0 & 0 & 0 & 0 & 1 \\
        \midrule
        Result & 0 & 0 & 0 & 0 & 0 & 0 & 0 & 0 \\
        \addlinespace
        \bottomrule
      \end{tabular}

    \end{adjustbox}

  }\caption*{X-OR'ing \$01 and \$01 gives \$00, the 'off' value for \icode{lineModeActivated}.}
\end{figure}
\clearpage

\textbf{Lines 1189-1231. \icode{\textbf{ApplyLineMode}}} 
\begin{lstlisting}[caption=From \icode{MainInterruptHandler}.]
;-------------------------------------------------------
; MainInterruptHandler
;-------------------------------------------------------
MainInterruptHandler
        ...
ApplyLineMode
        LDA lineModeActivated
        BEQ LineModeNotActive

        ; Line Mode Active
        LDA #NUM_ROWS + 1
        SEC 
        SBC cursorYPosition
        ORA #$80
        STA currentColorIndexArray,X
\end{lstlisting}

\begin{lstlisting}[caption=From \icode{MainPaintLoop}.]
;-------------------------------------------------------
; MainPaintLoop
;-------------------------------------------------------
MainPaintLoop    
        ...
        ; Line Mode sets the top bit of currentIndexToColorValues
        LDA baseLevelForCurrentPixel
        AND #$80   ; #LINE_MODE_ACTIVE
        BNE PaintLineModeAndLoop
        ...
PaintLineModeAndLoop
        ; Loops back to MainPaintLoop
        JMP PaintLineMode
\end{lstlisting}
\clearpage
\textbf{Lines 1189-1231. \icode{\textbf{ApplyLineMode}}:} Here \icode{lineModeActivated} is used
to taint the pixel buffers to indicate that line mode should be used for painting.

\textbf{Lines 1189-1231. \icode{\textbf{MainPaintLoop}}:} 
\clearpage
\textbf{Lines 1189-1231. \icode{\textbf{PaintLineMode}}} 
\begin{lstlisting}[caption=From \icode{PaintLineMode}.]
PaintLineMode 
        LDA currentIndexToColorValues
        AND #$7F
        STA offsetForYPos
        LDA #$19
        SEC 
        SBC offsetForYPos
        STA pixelYPositionZP
        DEC pixelYPositionZP
        LDA #$00
        STA currentIndexToColorValues
        LDA #$01
        STA skipPixel
        JSR PaintPixelForCurrentSymmetry
        INC pixelYPositionZP
        LDA #$00
        STA skipPixel

        LDA lineWidth
        EOR #$07
        STA currentIndexToColorValues
LineModeLoop   
        JSR PaintPixelForCurrentSymmetry
        INC pixelYPositionZP
        INC currentIndexToColorValues
        LDA currentIndexToColorValues
        CMP #$08
        BNE ResetLineModeColorValue
        JMP CleanUpAndExitLineModePaint

        INC currentIndexToColorValues
ResetLineModeColorValue   
        STA currentIndexToColorValues
        LDA pixelYPositionZP
        CMP #$19
        BNE LineModeLoop

CleanUpAndExitLineModePaint    
        LDX currentBufferLength
        DEC currentIndexForCurrentStepArray,X
        LDA currentIndexForCurrentStepArray,X
        CMP #$80
        BEQ ResetIndexAndExitLineModePaint
        JMP MainPaintLoop

ResetIndexAndExitLineModePaint   
        LDA #$FF
        STA currentIndexForCurrentStepArray,X
        STX shouldDrawCursor
        JMP MainPaintLoop
\end{lstlisting}
\clearpage

\textbf{Lines 1189-1231. \icode{\textbf{PaintLineMode}}:} 
\clearpage
