\begin{figure}[H]
    \centering
    \begin{adjustbox}{width=15cm,center}
      \includegraphics[width=10cm]{src/pulsespeed/pattern0-45.png}%
      \includegraphics[width=10cm]{src/pulsespeed/pattern1-45.png}%
    \end{adjustbox}
    \caption{Effect of low and high values for Pulse Speed}
\end{figure}
\begin{figure}[H]
    \centering
    \begin{adjustbox}{width=15cm,center}
      \includegraphics[width=10cm]{src/pulsewidth/pattern0-45.png}%
      \includegraphics[width=10cm]{src/pulsewidth/pattern1-45.png}%
    \end{adjustbox}
    \caption{Effect of low and high values for Pulse Width}
\end{figure}
\clearpage
\section*{pulse speed}
\label{sec:pulse_speed}
\lstset{style=6502Style}
\lstset{ 
   aboveskip=5pt,
   belowskip=0pt,
}

\begin{definition}[Jeffrey Says\index{Jeffrey Says}]
\setlength{\intextsep}{0pt}%
\setlength{\columnsep}{3pt}%
\begin{wrapfigure}{l}{0.12\textwidth}
\includegraphics[width=\linewidth]{src/callout/psych.png} 
\end{wrapfigure}
\small
\textbf{P to activate:} Usually if you hold down the button
you get a continuous stream. Setting the Pulse Speed allows you to
generate a pulsed stream, as if you were rapidly pressing and
releasing the FIRE button.
\\
\end{definition}

Pulsing is self explanatory enough, especially when you compare the
illustrations opposite. 

\section*{pulse width}
\begin{definition}[Jeffrey Says\index{Jeffrey Says}]
\setlength{\intextsep}{0pt}%
\setlength{\columnsep}{3pt}%
\begin{wrapfigure}{l}{0.12\textwidth}
\includegraphics[width=\linewidth]{src/callout/psych.png} 
\end{wrapfigure}
\small
\textbf{O to activate:} Sets the length of the pulses in a
pulsed stream output. Don’t worry about what that means - just get
in there and mess with it.
\\
\\
\end{definition}

Pulse speed and pulse width are managed together in the code as we shall
see. 

\clearpage
\textbf{Lines 1189-1231. \icode{\textbf{MaybePPressed}}} 
\begin{lstlisting}[caption=From \icode{CheckKeyboardInput\index{CheckKeyboardInput}}.,escapechar=\%]
MaybePPressed   
  CMP #KEY_P ; P pressed
  BNE MaybeHPressed

  ; P pressed.
  LDA #$04
  STA currentVariableMode%\index{currentVariableMode}%
  RTS 
\end{lstlisting}
\textbf{Lines 1189-1231. \icode{\textbf{UpdateVariableDisplay}}} 
\begin{lstlisting}[escapechar=\%][caption=From \icode{CheckKeyboardInputForActiveVariable}[escapechar=\%]. Pressing the \icode{<}[escapechar=\%] and > keys increments and
decrements the value in presetValueArray%\index{presetValueArray}% pointed to by \icode{X}\, i.e. \icode{currentVariableMode%\index{currentVariableMode}%}.]
UpdateVariableDisplay   
        ...

        LDX currentVariableMode%\index{currentVariableMode}%
        LDA lastKeyPressed%\index{lastKeyPressed}%
        CMP #KEY_GT ; > pressed?
        BNE MaybeLeftArrowPressed

RightArrowPressed
        ; > pressed, increase the value bar.
        INC presetValueArray%\index{presetValueArray}%,X
        LDA presetValueArray%\index{presetValueArray}%,X
        ; Make sure we don't exceed the max value.
        CMP maxValueForPresetValueArray,X
        BNE MaybeInColorMode%\index{MaybeInColorMode}%
        DEC presetValueArray%\index{presetValueArray}%,X
        JMP MaybeInColorMode%\index{MaybeInColorMode}%

MaybeLeftArrowPressed   
        CMP #KEY_LT ; < pressed?
        BNE MaybeInColorMode%\index{MaybeInColorMode}%

        ; < pressed, decrease the value bar.
        DEC presetValueArray%\index{presetValueArray}%,X
        LDA presetValueArray%\index{presetValueArray}%,X
        ; Make sure we don't exceed the min value.
        CMP minValueForPresetValueArray,X
        BNE MaybeInColorMode%\index{MaybeInColorMode}%
        INC presetValueArray%\index{presetValueArray}%,X
\end{lstlisting}
\clearpage

\textbf{Lines 1189-1231. \icode{\textbf{MaybePPressed}}:} When \icode{P} is pressed we don't
update a setting there and then as you might expect. Instead we get ready to display the 'Smoothing
Delay' control bar, by... loading the value \icode{\$04} to \icode{currentVariableMode\index{currentVariableMode}}? Okay, we'll
go with that for now.

\textbf{Lines 1189-1231. \icode{\textbf{UpdateVariableDisplay}}:}  The next time we loop around
to \icode{Check\-KeyboardInput}, we hit this little piece of logic at the very top of it:

\begin{lstlisting}[escapechar=\%]
CheckKeyboardInput%\index{CheckKeyboardInput}%   
        LDA currentVariableMode%\index{currentVariableMode}%
        BEQ CheckForGeneralKeystrokes
        JMP CheckKeyboardInputForActiveVariable
\end{lstlisting}

Well, we just loaded \icode{\$01} to \icode{currentVariableMode\index{currentVariableMode}} above so it's not zero.  It follows that
the \icode{BEQ} check will give a negative result (the check means 'is the value in \icode{A} equal to zero?'), 
so instead of forking to \icode{CheckForGeneralKeystrokes} we'll \icode{JMP} to \icode{CheckKeyboardInputForActiveVariable}.

This is where the function we're looking at here lives. As we can see the first thing it does is load 
\icode{currentVariableMode\index{currentVariableMode}} to the \icode{X} register. This is because we're going to use it as an index
into an array we encountered in the previous chapter on Presets. This is the array \icode{presetValueArray\index{presetValueArray}}
which contains a lot of the settings we'll be looking at in this chapter huddled together like a gaggle
of ducklings, with \icode{pulseSpeed} near at index 4 (index 0 being taken by \icode{unusedPresetByte}):

\begin{lstlisting}[escapechar=\%]
presetValueArray%\index{presetValueArray}%
unusedPresetByte        .BYTE $00
smoothingDelay%\index{smoothingDelay}%          .BYTE $0C
cursorSpeed             .BYTE $02
bufferLength%\index{bufferLength}%            .BYTE $1F
pulseSpeed              .BYTE $01 ; <-- Index $04
...
\end{lstlisting}

With \icode{X} set to 4 we can now use it increment the value for \icode{pulseSpeed} by
simply executing:
\begin{lstlisting}[escapechar=\%]
        INC presetValueArray%\index{presetValueArray}%,X
\end{lstlisting}
And decrement it by doing:
\begin{lstlisting}[escapechar=\%]
        DEC presetValueArray%\index{presetValueArray}%,X
\end{lstlisting}
This is handy, and worth remembering as the technique will be reused for adjusting other values that 
we look at that also live in \icode{presetValueArray\index{presetValueArray}}.

\clearpage

\clearpage
\begin{lstlisting}[caption=From \icode{MainInterruptHandler}.,escapechar=\%]
;-------------------------------------------------------
; MainInterruptHandler
;-------------------------------------------------------
MainInterruptHandler
        ...
        ; Player has pressed fire.
PlayerHasPressedFire   
        LDA stepsExceeded255
        BEQ DecrementPulseWidthCounter
        LDA stepsSincePressedFire%\index{stepsSincePressedFire}%
        BEQ IncrementStepsSincePressedFire
        JMP DrawCursorAndReturnFromInterrupt%\index{DrawCursorAndReturnFromInterrupt}%

IncrementStepsSincePressedFire   
        INC stepsSincePressedFire%\index{stepsSincePressedFire}%

DecrementPulseWidthCounter   
        LDA currentPulseWidth%\index{currentPulseWidth}%
        BEQ DecrementPulseSpeedCounter%\index{DecrementPulseSpeedCounter}%
        DEC currentPulseWidth%\index{currentPulseWidth}%
        BEQ DecrementPulseSpeedCounter%\index{DecrementPulseSpeedCounter}%
        JMP UpdatePixelBuffersForPattern

DecrementPulseSpeedCounter%\index{DecrementPulseSpeedCounter}%   
        DEC currentPulseSpeedCounter%\index{currentPulseSpeedCounter}%
        BEQ RefreshPulseSpeedAndWidth
        JMP DrawCursorAndReturnFromInterrupt%\index{DrawCursorAndReturnFromInterrupt}%

RefreshPulseSpeedAndWidth   
        LDA pulseSpeed
        STA currentPulseSpeedCounter%\index{currentPulseSpeedCounter}%
        LDA pulseWidth
        STA currentPulseWidth%\index{currentPulseWidth}%

        ; Finally, update the pixel buffers with a byte
        ; each for the current pattern.        
UpdatePixelBuffersForPattern    
        INC currentStepCount%\index{currentStepCount}%
        LDA currentStepCount%\index{currentStepCount}%
        CMP bufferLength%\index{bufferLength}%
        BNE UpdateBaseLevelArray

        LDA #$00
        STA currentStepCount%\index{currentStepCount}%
\end{lstlisting}
\clearpage

\textbf{Lines 1189-1231. \icode{\textbf{MainInterruptHandler}}:} 
As you may remember the \icode{Main\-Interrupt\-Handler} runs every 1/60th of a
second.  You may also recall its main job is to fill the pixel buffers (e.g.
pixelXPositionArray\index{pixelXPositionArray}, pixelYPositionArray\index{pixelYPositionArray} and so on) so that the MainPaintLoop\index{MainPaintLoop}
can use them to paint the screen. 

\textbf{Lines 1189-1231. \icode{\textbf{DecrementPulseWidthCounter}}:} 
Pulse speed and width both act as counters, each defining an interval. Pulse speed
defines an interval that controls how often the buffers are refreshed, while 
pulse width effectivaly acts as a multiplier for pulse speed.

We can see this when we observe that we decrement pulse width first and it is only
when \icode{currentPulseWidth\index{currentPulseWidth}} reaches zero that we consider decrementing pulse speed.

Meanwhile when we do decrement pulse speed (effectively every pulse width reaches zero)
we will always bail out without updating our pixel buffers until \icode{current\-Pulse\-SpeedCounter}
reaches zero.

\clearpage

