\chapter{colourflow} 
\label{sec:presets}
\lstset{style=6502Style}
\begin{definition}[Jeffrey Says]
\setlength{\intextsep}{0pt}%
\setlength{\columnsep}{3pt}%
\begin{wrapfigure}{l}{0.12\textwidth}
\includegraphics[width=\linewidth]{src/callout/psych.png} 
\end{wrapfigure}
\small
Altering All Colours Simultaneously

This is useful if you have a colourflow in, for example, shades of blue, and
you wish to change to greens or reds.  It is possible to change each colour in
the flow individually (as detailed later), but when each colour has only to be
transposed by an equal amount, it is simpler to change all the colours
simultaneously.

In order to understand how best to use the colour controls, you need to know
how the colours are numbered.  There are sixteen basic hues, and sixteen
intensity levels within each hue.  For example, colour 0 is black and colour 15
is white.  The numbers in between represent the intermediate shades of grey
between black and white.  It's easier to think in terms of music:  steps of 16
are like octaves in music.  If you add 16 to a particular colour, you get the
next colour up in the same intensity.  The colours themselves are laid out in
more or less rainbow order.

To alter all the colours at once, you use a variable known as the Simultaneous
Adder.  Initially, this is set to 16, but you can adjust it to whatever you
like (see Adjusting a Variable, below).  Try adjusting the colourflow by
performing the following actions:

Press * until the top of the screen displays "CKEYS=COLOURS".  This tells the
system that you will be operating on the colour values.  Then press the "H" key
(Perform Simultaneous Add).  The message "COLOURFLOW RESYNCED" will appear, and
if you make some patterns you will see that the colours have changed slightly.
  The more you press "H" the more the colourflow is transposed.

\end{definition}
\clearpage

\subfile{colorspace_colourflow/flow1.tex}
\clearpage
Before we take a look at how it's achieved, we can see on the opposite page
some example of colourflow in action.
\clearpage
\textbf{Lines 1189-1231. \icode{\textbf{HPressed}}} 
\begin{lstlisting}
;----------------------------------------
; HPressed   
;----------------------------------------
HPressed   
        LDA #$00
        LDY currentColourControlEffect
        BEQ NoCurrentColorControlEffect

AddOffsetToOozeArray
_Loop   CLC 
        ADC #$08
        DEY 
        BNE _Loop

NoCurrentColorControlEffect   
				TAX 
        PLA 
        AND #$40
        BEQ UpdateColorValues

ClearColourValues
        INX 
        LDY #$07
ClearClrValuesLoop   
        LDA #$00
        STA colorPallette,X
        INX 
        DEY 
        BNE ClearClrValuesLoop
        JMP ColourFlowResync

UpdateColorValues   
				LDY #$07
        INX 
UpdateColorValuesLoop
        LDA colorPallette,X
        CLC 
        ADC simlAdder
        STA colorPallette,X
        INX 
        DEY 
        BNE UpdateColorValuesLoop

        JMP ColourFlowResync
        ; Returns


\end{lstlisting}
\clearpage

\textbf{Lines 1189-1231. \icode{\textbf{HPressed}}:} 
text
\clearpage

\textbf{Lines 1189-1231. \icode{\textbf{ColourFlowResync}}} 
\begin{lstlisting}
COLOURFLOW_RESYNCHED = $1F
;------------------------------------------------------
; ColourFlowResync
;------------------------------------------------------
ColourFlowResync
        LDX #$00
_Loop   
        LDA colorPallette,X
        STA PCOLR0,X ;PCOLR0  shadow for COLPM0 ($D012)

        LDA oozeCycles,X
        STA oozeCyclesCounter,X

        LDA oozeRates,X
        STA oozeRateCounter,X

        LDA #$00
        STA oozeRateTracker,X

        INX 
        CPX #$08
        BNE _Loop

        LDX #COLOURFLOW_RESYNCHED
        JMP UpdateStatusLine
        ; Returns

\end{lstlisting}
\textbf{Lines 1189-1231. \icode{\textbf{ColourFlowResync}}} 
\begin{lstlisting}
colorPallette              .BYTE $00,$18,$38,$58,$78,$98,$B8,$D8
oozeRates                  .BYTE $00,$00,$00,$00,$00,$00,$00,$00
oozeSteps                  .BYTE $01,$01,$01,$01,$01,$01,$01,$01
oozeCycles                 .BYTE $0F,$0F,$0F,$0F,$0F,$0F,$0F,$0F
oozeRateCounter            .BYTE $00,$00,$00,$00,$00,$00,$00,$00
oozeCyclesCounter          .BYTE $0F,$0F,$0F,$0F,$0F,$0F,$0F,$0F
oozeRateTracker            .BYTE $00,$00,$00,$00,$00,$00,$00,$00
\end{lstlisting}
\clearpage

\textbf{Lines 1189-1231. \icode{\textbf{ColourFlowResync}}:} 
text
\clearpage
\subfile{colorspace_colourflow/flow2.tex}
\clearpage
\textbf{Lines 1189-1231. \icode{\textbf{UpdateStatusLine}}} 
\begin{lstlisting}
;------------------------------------------------------
; UpdateStatusLine
;------------------------------------------------------
UpdateStatusLine
        LDA #<statusLineText
        STA statusLineTextLoByte

        LDA #>statusLineText
        STA statusLineTextHiByte

        LDA disableStatusLine
        BEQ StatusLineEnabled
        RTS 

StatusLineEnabled   
        CPX #$00
        BEQ StoreStatusLinePtrs
_Loop   LDA statusLineTextLoByte
        CLC 
        ADC #$14
        STA statusLineTextLoByte

        LDA statusLineTextHiByte
        ADC #$00
        STA statusLineTextHiByte

        DEX 
        BNE _Loop

StoreStatusLinePtrs   
        LDA statusLineTextLoByte
        STA statusLineLoPtr

        LDA statusLineTextHiByte
        STA statusLineHiPtr

        LDA #$30
        STA counterTillNextKeyboardInput

        RTS 

\end{lstlisting}
\textbf{Lines 1189-1231. \icode{\textbf{statusLineText}}} 
\begin{lstlisting}
statusLineText
        ...
        .TEXT 'C.KEYS:   OOZE STEPS'
        .TEXT 'C.KEYS:  OOZE CYCLES'
        .TEXT 'COLOURFLOW RESYNCHED'
        .TEXT 'PULSE FLOW RATE: 000'
        ...
\end{lstlisting}
\clearpage

\textbf{Lines 1189-1231. \icode{\textbf{statusLineText}}:} 
text
\clearpage
\textbf{Lines 1189-1231. \icode{\textbf{GenerateDisplayList}}} 
\begin{lstlisting}
;------------------------------------------------------
; GenerateDisplayList
;------------------------------------------------------
GenerateDisplayList
        LDA #>displayListInstructions
        STA displayListInstructionsHiPtr
        LDA #<displayListInstructions
        STA displayListInstructionsLoPtr
        LDA #>foregroundPixelsOriginalLocation
        STA foregroundPixelsHiPtr
        LDA #<foregroundPixelsOriginalLocation
        STA foregroundPixelsLoPtr

        LDX numberOfLinesToDraw
        LDA totalNumberOfLinesToDrawArray,X
        STA bottomMostLineNumber

        LDY #$00
        STY bottomMostYPos

        LDA #$70 ; 8 blank lines
        JSR WriteValueToDisplayList
        JSR WriteValueToDisplayList

        ; Mode 6 setting screen memory to statusLineLoPtr
        ; and statusLineHiPtr
        LDA #$46
        JSR WriteValueToDisplayList
        LDA statusLineLoPtr
        JSR WriteValueToDisplayList
        LDA statusLineHiPtr
        JSR WriteValueToDisplayList

\end{lstlisting}
\clearpage

\textbf{Lines 1189-1231. \icode{\textbf{GenerateDisplayList}}:} 
text
\clearpage
\captionsetup[figure]{font=tiny}
\subfile{colorspace_colourflow/flows.tex}
