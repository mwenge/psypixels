\chapter{increasing the dose} 
\label{sec:fullgame}
\rhead[]{\leftmark}
\lstset{style=6502Style}
\lstset{ 
   aboveskip=5pt,
   belowskip=0pt,
}
\begin{definition}[Jeffrey Says]
\setlength{\intextsep}{0pt}%
\setlength{\columnsep}{3pt}%
\begin{wrapfigure}{l}{0.12\textwidth}
\includegraphics[width=\linewidth]{src/callout/psych.png} 
\end{wrapfigure}
\small
  I felt that something so basic and lovely
  deserved more than just being another thing to be sold and profited from. I
  actually gave that first algorithm away in listing form to a computer
  magazine. But my parents argued successfully that there was no shame in
  making some money from it, and so in due course I created a somewhat expanded
  version with more patterns and control options, and that is what was released
  as Psychedelia.
\end{definition}

The magazine listing was compact enough to reduce to two pages. That is not the case
with the commercial edition of Psychedelia which runs to almost sixteen thousand lines of
code and creates a binary of 10,000 or so bytes. This is larger, but still relatively small. Since the rest of this book will 
anatomize the interesting features of the commercial game it makes sense for us to
begin with a similar attempt at providing an atlas to the code base as a whole.

As before, we're going to set out the full listing so that we get an initial sense of the
size of the program and an understanding of the order in which the features of the game
and the data that supports them are implemented.

\subfile{listing_commentary/full_psychedelia_listing.tex}
\clearpage
