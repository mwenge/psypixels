\chapter{all the pretty patterns} 
\label{sec:patterns}
\rhead[]{\leftmark}
\lstset{style=6502Style}
\lstset{ 
   aboveskip=5pt,
   belowskip=0pt,
}

\begin{definition}[Jeffrey Says]
\setlength{\intextsep}{0pt}%
\setlength{\columnsep}{3pt}%
\begin{wrapfigure}{l}{0.12\textwidth}
\includegraphics[width=\linewidth]{src/callout/psych.png} 
\end{wrapfigure}
\small
Choose a pattern you like and get ready to
experiment. Press S to change the Symmetry. (The pattern gets
reflected in various planes, or not at all according to the setting).
Press SPACE to alter the pattern element. There are eight
permanent ones, and eight you can define for yourself (more on
this later!). The latter eight are all set up when you load, so you can
always choose from 16 shapes.
\end{definition}

We deserve a rest from reading so much code. So let's take a brief intermission where
we admire all the different patterns that Psychedelia is packaged with. On each of the following
pages we give a 3D rendering of each pattern's evolution according to the
\hyperref[sec:listing_pattern]{\textcolor{blue}{'Sunday Run Algorithm'}}.
On the facing page we provide the two data arrays used to generate the pattern and a step by step
visualisation of the pattern's evolution.

The bad news is that there is some extra code for us to cover from the commercial edition, not much,
but you will find it sprinkled among the pretty pictures and it will hopefully shed some light
on how the commercial edition of Psychedelia managed the large number of patterns available to the
player.

\clearpage
In order to enjoy the pretty pictures it might help to understand what they mean. We're already 
familiar with the data structure that underlies the patterns. Here is the structure for the Star
Pattern again, a pair of arrays: the first giving the X co-ordinates, the second giving the Y
co-ordinates.
\begin{lstlisting}[caption=Source code for the Star.]
starOneXPosArray  
  .BYTE $00,$01,$01,$01,$00,$FF,$FF,$FF,$55       ;        5       
  .BYTE $00,$02,$00,$FE,$55                       ;                
  .BYTE $00,$03,$00,$FD,$55                       ;       4 4      
  .BYTE $00,$04,$00,$FC,$55                       ;        3       
  .BYTE $FF,$01,$05,$05,$01,$FF,$FB,$FB,$55       ;        2       
  .BYTE $00,$07,$00,$F9,$55                       ;        1       
  .BYTE $55                                       ;   4   000   4  
starOneYPosArray                                  ; 5  3210 0123  5  
  .BYTE $FF,$FF,$00,$01,$01,$01,$00,$FF,$55       ;   4   000   4  
  .BYTE $FE,$00,$02,$00,$55                       ;        1       
  .BYTE $FD,$00,$03,$00,$55                       ;        2       
  .BYTE $FC,$00,$04,$00,$55                       ;        3       
  .BYTE $FB,$FB,$FF,$01,$05,$05,$01,$FF,$55       ;       4 4      
  .BYTE $F9,$00,$07,$00,$55                       ;                
  .BYTE $55                                       ;        5       
\end{lstlisting}

For each of the patterns I attempt to visualize its evolution using a table such as this one for 
the Star One pattern:

\begin{figure}[H]
    \centering
    \begin{adjustbox}{width=10cm,center}
      \footnotesize
      \begin{tabular}{ll}

        \makecell[l]{
\icode{.BYTE \$00,\$01,\$01,\$01,\$00,\$FF,\$FF,\$FF}\\
\icode{.BYTE \$FF,\$FF,\$00,\$01,\$01,\$01,\$00,\$FF}
} & \makecell[l]{
\includegraphics[width=1.3cm]{src/patterns/pixels/pixel_pattern0_0.png}%
} \\
        \midrule

        \makecell[l]{
\icode{.BYTE \$00,\$02,\$00,\$FE}\\
\icode{.BYTE \$FE,\$00,\$02,\$00}
} & \makecell[l]{
\includegraphics[width=1.3cm]{src/patterns/pixels/pixel_pattern0_1.png}%
\includegraphics[width=1.3cm]{src/patterns/pixels/pixel_pattern0_2.png}%
} \\
        \midrule

        \makecell[l]{
\icode{.BYTE \$00,\$03,\$00,\$FD}\\
\icode{.BYTE \$FD,\$00,\$03,\$00}
} & \makecell[l]{
\includegraphics[width=1.3cm]{src/patterns/pixels/pixel_pattern0_3.png}%
\includegraphics[width=1.3cm]{src/patterns/pixels/pixel_pattern0_4.png}%
\includegraphics[width=1.3cm]{src/patterns/pixels/pixel_pattern0_5.png}%
} \\
        \midrule

        \makecell[l]{
\icode{.BYTE \$00,\$04,\$00,\$FC}\\
\icode{.BYTE \$FC,\$00,\$04,\$00}
} & \makecell[l]{
\includegraphics[width=1.3cm]{src/patterns/pixels/pixel_pattern0_6.png}%
\includegraphics[width=1.3cm]{src/patterns/pixels/pixel_pattern0_7.png}%
\includegraphics[width=1.3cm]{src/patterns/pixels/pixel_pattern0_8.png}%
\includegraphics[width=1.3cm]{src/patterns/pixels/pixel_pattern0_9.png}%
} \\
        \midrule

        \makecell[l]{
\icode{.BYTE \$FF,\$01,\$05,\$05,\$01,\$FF,\$FB,\$FB}\\
\icode{.BYTE \$FB,\$FB,\$FF,\$01,\$05,\$05,\$01,\$FF}
} & \makecell[l]{
\includegraphics[width=1.3cm]{src/patterns/pixels/pixel_pattern0_10.png}%
\includegraphics[width=1.3cm]{src/patterns/pixels/pixel_pattern0_11.png}%
\includegraphics[width=1.3cm]{src/patterns/pixels/pixel_pattern0_12.png}%
\includegraphics[width=1.3cm]{src/patterns/pixels/pixel_pattern0_13.png}%
\includegraphics[width=1.3cm]{src/patterns/pixels/pixel_pattern0_14.png}%
} \\
        \midrule

        \makecell[l]{
\icode{.BYTE \$00,\$07,\$00,\$F9}\\
\icode{.BYTE \$F9,\$00,\$07,\$00}
} & \makecell[l]{
\includegraphics[width=1.3cm]{src/patterns/pixels/pixel_pattern0_15.png}%
\includegraphics[width=1.3cm]{src/patterns/pixels/pixel_pattern0_16.png}%
\includegraphics[width=1.3cm]{src/patterns/pixels/pixel_pattern0_17.png}%
\includegraphics[width=1.3cm]{src/patterns/pixels/pixel_pattern0_18.png}%
\includegraphics[width=1.3cm]{src/patterns/pixels/pixel_pattern0_19.png}%
\includegraphics[width=1.3cm]{src/patterns/pixels/pixel_pattern0_20.png}%
} \\
        \midrule

          \end{tabular}
    \end{adjustbox}
\end{figure}

In order to understand what this table is telling us, let's look at the first entry more
closely.
\begin{figure}[H]
  \centering
  \begin{adjustbox}{width=8cm,center}
    \footnotesize
    \begin{tabular}{ll}

      \makecell[l]{
        \icode{.BYTE \$00,\$01,\$01,\$01,\$00,\$FF,\$FF,\$FF}\\
        \icode{.BYTE \$FF,\$FF,\$00,\$01,\$01,\$01,\$00,\$FF}
      } & \makecell[l]{
        \includegraphics[width=1.3cm]{src/patterns/pixels/pixel_pattern0_0.png}%
      } \\
    \end{tabular}
  \end{adjustbox}
\end{figure}

This is the first line in \icode{starOneXPosArray} and the first line in \icode{starOneYPosArray} listed
together. It's followed by how these pairs of co-ordinates are plotted onto the screen. If you're paying close
attention you'll notice that the \icode{\$55} at the end of each line has been dropped. That's because the
\icode{\$55} is a 'sentinel' value that tells the 'Sunday Run' algorithm to treat this as a kind of line-break
and draw the preceding co-ordinates a single iteration of the paint loop.

The true puzzle is how values like \icode{\$FF} and \icode{\$00} can consitute co-ordinates! The answer to this
is illustrated in the table below where we show how this first pair of lines from \icode{starOneXPosArray} and
\icode{starOneYPosArray} is plotted. The numbers given in the middle of the table represent the order that the
pairs appear in each array. 

\subfile{patterns/pattern0_array_table_first.tex}

So for example, the first pair from each array (\icode{\$00} and \icode{\$FF}) is given as \icode{1} above. The
second pair (\icode{\$01} and \icode{\$FF}) is given as \icode{2}. So as you can hopefully infer: \icode{\$FF}
is our way of saying '-1' in this co-ordinate system, \icode{\$FE} our way of saying '-2' and so on. This is the
conventional way of representing or treating single byte values when we want to represent both positive and
negative numbers. Conventionally, since a single byte can represent 256 different values, if we want it to contain
both positive and negative numbers we can use a single byte to represent anything between -128 and +128. 

If we look at the second row in our table we've moved on to the co-ordinates given by the second line in each of
\icode{starOneXPosArray} and \icode{starOneYPosArray}:

\begin{figure}[H]
  \centering
  \begin{adjustbox}{width=8cm,center}
    \footnotesize
    \begin{tabular}{ll}

      \makecell[l]{
        \icode{.BYTE \$00,\$02,\$00,\$FE}\\
        \icode{.BYTE \$FE,\$00,\$02,\$00}
      } & \makecell[l]{
        \includegraphics[width=1.3cm]{src/patterns/pixels/pixel_pattern0_1.png}%
        \includegraphics[width=1.3cm]{src/patterns/pixels/pixel_pattern0_2.png}%
      } \\
    \end{tabular}
  \end{adjustbox}
\end{figure}

Here we plot in the four pixels given by the lines. We've added them in gray below. 

\subfile{patterns/pattern0_array_table_second.tex}

The two pictures on the right hand side of the table above show us layering in this extra set of pixels. But they also show us something else:
the way Psychedelia is cycling the colors of the pixels while painting the structure at this stage of its evolution.

Finally let's look at all the entries in our two arrays plotted on the same graph. In this instance we use numbers in the
plot to represent the line in the array they come from (and give each a different color too to aid identification). So
for example the points plotted by the first line in \icode{starOneXPosArray} and \icode{starOneYPosArray} are shown 
in green with the value \icode{0}. These colors and values are just to aid apprehensions - they have no other meaning.

As we can see each 'line' the arrays builds up the evolution of the pattern. The purpose of the tables on the following
pages is to help you understand how the pattern is built up in each case as well as to show how the Sunday Run algorithm
colours the pixels during each painting iteration. 

\subfile{patterns/pattern0_array_table.tex}

In addition to tabulating the pattern eveolution in this way, we also visualize the evolution of each pattern in three dimensions. Hopefully it is obvious how these pictures 
represent the evolution of the pattern over time. In each case we view the evolution from both the 'front' and the 'back'.
They show how the coloring of the pixels for each portion of the pattern changes over time as well as how the pattern
emerges while it is constructed.
\begin{figure}[H]
    \centering
    \begin{adjustbox}{width=9cm,center}
      \includegraphics[width=5cm]{src/patterns/pattern0-45.png}%
      \includegraphics[width=5cm]{src/patterns/pattern0-225.png}%
    \end{adjustbox}
\caption{Evolution of the 'Star One' pattern.}
\end{figure}

As we said earlier there's some code and commentary sprinkled across the following pages that is relevant to the drawing
of these patterns. Hopefully it will continue to shed more light on the operation of Psychedelia.

\clearpage
\begin{figure}[H]
    \centering
    \begin{adjustbox}{width=12cm,center}
      \includegraphics[width=12cm]{src/patterns/pattern0-45.png}%
    \end{adjustbox}
    \begin{adjustbox}{width=12cm,margin=0cm -2cm}
      \includegraphics[width=12cm]{src/patterns/pattern0-225.png}%
    \end{adjustbox}
\caption{Evolution of the 'Star One' pattern.}
\end{figure}

\rhead[]{Star One}
\begin{lstlisting}[caption=Source code for the Star.]
starOneXPosArray  
  .BYTE $00,$01,$01,$01,$00,$FF,$FF,$FF,$55       ;        5       
  .BYTE $00,$02,$00,$FE,$55                       ;                
  .BYTE $00,$03,$00,$FD,$55                       ;       4 4      
  .BYTE $00,$04,$00,$FC,$55                       ;        3       
  .BYTE $FF,$01,$05,$05,$01,$FF,$FB,$FB,$55       ;        2       
  .BYTE $00,$07,$00,$F9,$55                       ;        1       
  .BYTE $55                                       ;   4   000   4  
starOneYPosArray                                  ; 5  3210 0123  5  
  .BYTE $FF,$FF,$00,$01,$01,$01,$00,$FF,$55       ;   4   000   4  
  .BYTE $FE,$00,$02,$00,$55                       ;        1       
  .BYTE $FD,$00,$03,$00,$55                       ;        2       
  .BYTE $FC,$00,$04,$00,$55                       ;        3       
  .BYTE $FB,$FB,$FF,$01,$05,$05,$01,$FF,$55       ;       4 4      
  .BYTE $F9,$00,$07,$00,$55                       ;                
  .BYTE $55                                       ;        5       
\end{lstlisting}

\subfile{patterns/tables/pattern0.tex}


\clearpage
\textbf{Lines 1189-1231. \icode{\textbf{pixelXPositionLoPtrArray, pixelYPositionLoPtrArray}}} 
\begin{lstlisting}[caption = All the pattern data structures in Psychedelia organized into a set of arrays. 
We use these arrays to choose the correct user-selected pattern at painting time.]
; A pair of arrays together consituting a list of pointers
; to positions in memory containing X position data.
; (i.e. $097C, $0E93,$0EC3, $0F07, $0F23, $0F57, $1161, $11B1)
pixelXPositionLoPtrArray
   .BYTE <starOneXPosArray,<theTwistXPosArray,<laLlamitaXPosArray
   .BYTE <starTwoXPosArray,<deltoidXPosArray,<diffusedXPosArray
   .BYTE <multicrossXPosArray,<pulsarXPosArray
   .BYTE <customPattern0XPosArray,<customPattern1XPosArray
   .BYTE <customPattern2XPosArray,<customPattern3XPosArray
   .BYTE <customPattern4XPosArray,<customPattern5XPosArray
   .BYTE <customPattern6XPosArray,<customPattern7XPosArray

pixelXPositionHiPtrArray 
   .BYTE >starOneXPosArray,>theTwistXPosArray,>laLlamitaXPosArray
   .BYTE >starTwoXPosArray,>deltoidXPosArray,>diffusedXPosArray
   .BYTE >multicrossXPosArray,>pulsarXPosArray
   .BYTE >customPattern0XPosArray,>customPattern1XPosArray
   .BYTE >customPattern2XPosArray,>customPattern3XPosArray
   .BYTE >customPattern4XPosArray,>customPattern5XPosArray
   .BYTE >customPattern6XPosArray,>customPattern7XPosArray


; A pair of arrays together consituting a list of pointers
; to positions in memory containing Y position data.
; (i.e. $097C, $0E93,$0EC3, $0F07, $0F23, $0F57, $1161, $11B1)
pixelYPositionLoPtrArray 
   .BYTE <starOneYPosArray,<theTwistYPosArray,<laLlamitaYPosArray
   .BYTE <starTwoYPosArray,<deltoidYPosArray,<diffusedYPosArray
   .BYTE <multicrossYPosArray,<pulsarYPosArray
   .BYTE <customPattern0YPosArray,<customPattern1YPosArray
   .BYTE <customPattern2YPosArray,<customPattern3YPosArray
   .BYTE <customPattern4YPosArray,<customPattern5YPosArray
   .BYTE <customPattern6YPosArray,<customPattern7YPosArray

pixelYPositionHiPtrArray 
   .BYTE >starOneYPosArray,>theTwistYPosArray,>laLlamitaYPosArray
   .BYTE >starTwoYPosArray,>deltoidYPosArray,>diffusedYPosArray
   .BYTE >multicrossYPosArray,>pulsarYPosArray
   .BYTE >customPattern0YPosArray,>customPattern1YPosArray
   .BYTE >customPattern2YPosArray,>customPattern3YPosArray
   .BYTE >customPattern4YPosArray,>customPattern5YPosArray
   .BYTE >customPattern6YPosArray,>customPattern7YPosArray

\end{lstlisting}
\clearpage

\rhead[]{\icode{pixelXPositionLoPtrArray/pixelXPositionHiPtrArray}}
\textbf{Lines 1189-1231. \icode{\textbf{pixelXPositionLoPtrArray/pixelXPositionHiPtrArray}}:} Psychedelia
offers 16 different pretty patterns to choose from, so requires some way of managing them, particularly
when it comes time to painting them on the screen. The four arrays on the opposite page do this job.
They allow us to reference each pattern with an index. For example, index 0 will reference the X and
Y Position data structures for the 'Star One' pattern in \icode{starOneXPosArray} and 
\icode{starOneYPosArray}, index 1 will allow us to reference the data structures for 'The Twist' pattern,
and so on.

On the following pages we'll see how we make practical use of these arrays, but for now we only really
need to make clear that each one contains one byte of the two-byte address at which each
data structure is stored. The use of \icode{<} and \icode{>} in the listing is a convention that
tells the assembler we're looking at the first byte (\icode{>}) or the second byte (\icode{<}).
Hopefully this table makes this explicit to the reader:

\begin{figure}[H]
  {
    \setlength{\tabcolsep}{3.0pt}
    \setlength\cmidrulewidth{\heavyrulewidth} % Make cmidrule = 
    \begin{adjustbox}{width=8cm,center}
      \begin{tabular}{ccccc}
        \toprule
        Element &
        \makecell[c]{\icode{pixelXPosition} \\ \icode{HiPtrArray}} & 
        \makecell[c]{\icode{pixelXPosition} \\ \icode{LoPtrArray}} & 
        Address &
        Name \\
        \midrule
        0 & \icode{\$09} & \icode{\$7C} & \icode{\$097C} & \icode{starOneXPosArray} \\ 
        1 & \icode{\$0E} & \icode{\$93} & \icode{\$0E93}  & \icode{theTwistXPosArray}\\ 
        2 & \icode{\$0E} & \icode{\$C3} & \icode{\$0EC3}  & \icode{laLlamitaXPosArray}\\ 
        3 & \icode{\$0F} & \icode{\$07} & \icode{\$0F07}  & \icode{starTwoXPosArray}\\ 
        4 & \icode{\$0F} & \icode{\$23} & \icode{\$0F23}  & \icode{deltoidXPosArray}\\ 
        5 & \icode{\$0F} & \icode{\$57} & \icode{\$0F57}  & \icode{diffusedXPosArray}\\ 
        . & . & . & . &. \\
        15 & \icode{\$CE} & \icode{\$00} & \icode{\$CE00}  & \icode{customPattern6XPosArray}\\ 
        16 & \icode{\$CF} & \icode{\$00} & \icode{\$CF00}  & \icode{customPattern7XPosArray}\\ 
        \bottomrule
      \end{tabular}
    \end{adjustbox}
  }\caption{Our two arrays and their contents - each combining to give us an address for the X
  Position data structure for each pattern. The line with ellipses indicates that we've left out some elements for the
  sake of brevity.}
\end{figure}
\vspace*{-\baselineskip}

\textbf{Lines 1189-1231. \icode{\textbf{pixelYPositionLoPtrArray/pixelYPositionHiPtrArray}}:} As for the X position array,
so for the Y position array. 
\begin{figure}[H]
  {
    \setlength{\tabcolsep}{3.0pt}
    \setlength\cmidrulewidth{\heavyrulewidth} % Make cmidrule = 
    \begin{adjustbox}{width=8cm,center}
      \begin{tabular}{ccccc}
        \toprule
        Element &
        \makecell[c]{\icode{pixelYPosition} \\ \icode{HiPtrArray}} & 
        \makecell[c]{\icode{pixelYPosition} \\ \icode{LoPtrArray}} & 
        Address &
        Name \\
        \midrule
        0 & \icode{\$09} & \icode{\$A3} & \icode{\$09A3} & \icode{starOneYPosArray} \\ 
        1 & \icode{\$0E} & \icode{\$AB} & \icode{\$0EAB}  & \icode{theTwistYPosArray}\\ 
        2 & \icode{\$0E} & \icode{\$E5} & \icode{\$0EE5}  & \icode{laLlamitaYPosArray}\\ 
        . & . & . & . &. \\
        15 & \icode{\$CE} & \icode{\$00} & \icode{\$CE00}  & \icode{customPattern6YPosArray}\\ 
        16 & \icode{\$CF} & \icode{\$00} & \icode{\$CF00}  & \icode{customPattern7YPosArray}\\ 
        \bottomrule
      \end{tabular}
    \end{adjustbox}
  }\caption{Our two arrays and their contents - each combining to give us an address for the Y
  Position data structure for each pattern. }
\end{figure}
\clearpage
\begin{figure}[H]
    \centering
    \begin{adjustbox}{width=12cm,center}
      \includegraphics[width=12cm]{src/patterns/pattern1-45.png}%
    \end{adjustbox}
    \begin{adjustbox}{width=12cm,margin=0cm -4cm}
      \includegraphics[width=12cm]{src/patterns/pattern1-225.png}%
    \end{adjustbox}
\caption{The 'Twist'.}
\end{figure}
\clearpage

\rhead[]{The Twist}
\begin{lstlisting}
theTwistXPosArray .BYTE $00,$55                            ;     1  
                  .BYTE $01,$02,$55                        ;   01   
                  .BYTE $01,$02,$03,$55                    ;  6 222 
                  .BYTE $01,$02,$03,$04,$55                ;  543   
                  .BYTE $00,$00,$00,$55                    ; 5 4 3  
                  .BYTE $FF,$FE,$55                        ;   4  3 
                  .BYTE $FF,$55                            ;       3
                  .BYTE $55
theTwistYPosArray .BYTE $FF,$55
                  .BYTE $FF,$FE,$55
                  .BYTE $00,$00,$00,$55
                  .BYTE $01,$02,$03,$04,$55
                  .BYTE $01,$02,$03,$55
                  .BYTE $01,$02,$55
                  .BYTE $00,$55
                  .BYTE $55
\end{lstlisting}
\subfile{patterns/tables/pattern1.tex}

\clearpage
\textbf{Lines 1189-1231. \icode{\textbf{PaintStructureAtCurrentPosition}}} 
\begin{lstlisting}[basicstyle=\ttfamily\scriptsize, caption=The routine responsible for painting patterns.]
xPosLoPtr = $0D
xPosHiPtr = $0E
yPosLoPtr = $10
yPosHiPtr = $11
;-------------------------------------------------------
; PaintStructureAtCurrentPosition
;-------------------------------------------------------
PaintStructureAtCurrentPosition   
        JSR PaintPixelForCurrentSymmetry
        LDY #$00
        LDA baseLevelForCurrentPixel
        CMP #$07
        BNE CanLoopAndPaint
        RTS 

CanLoopAndPaint   
        LDA #$07
        STA countToMatchCurrentIndex

        LDA pixelXPosition
        STA previousCursorXPosition
        LDA pixelYPosition
        STA previousPixelYPosition

        LDX patternIndex
        LDA pixelXPositionLoPtrArray,X
        STA xPosLoPtr
        LDA pixelXPositionHiPtrArray,X
        STA xPosHiPtr
        LDA pixelYPositionLoPtrArray,X
        STA yPosLoPtr
        LDA pixelYPositionHiPtrArray,X
        STA yPosHiPtr

        ; Paint pixels in the sequence until hitting a break
        ; at $55
PixelPaintLoop   
        ; Get the next byte from the pattern's X pos array.
        LDA previousCursorXPosition
        CLC 
        ADC (xPosLoPtr),Y
        STA pixelXPosition

        ; Get the next byte from the pattern's Y pos array.
        LDA previousPixelYPosition
        CLC 
        ADC (yPosLoPtr),Y
        STA pixelYPosition

        ; Push Y to the stack.
        TYA 
        PHA 

        JSR PaintPixelForCurrentSymmetry

        ; Pull Y back from the stack and increment
        PLA 
        TAY 
        INY 
\end{lstlisting}
\clearpage

\rhead[]{\icode{PaintStructureAtCurrentPosition}}
\textbf{Lines 1189-1231. \icode{\textbf{PaintStructureAtCurrentPosition}}:} We've already encountered this routine
in \hyperref[sec:listing_commentary]{\textcolor{blue}{ our walk through of the listing.}}. This is the version that shipped 
with the commercial edition of Psychedelia so has a necessary extra complication to deal with the fact that we 
are going to paint one of up to 16 possible patterns. What we want to figure out here is the X and Y position we should
paint for each element in the pattern's data structure.

The pattern the user has selected is stored as an index value in \icode{patternIndex}. We use it to fetch the element
from each array:
\begin{lstlisting}[basicstyle=\ttfamily\scriptsize]
        LDX patternIndex
        LDA pixelXPositionLoPtrArray,X
        STA xPosLoPtr
        LDA pixelXPositionHiPtrArray,X
        STA xPosHiPtr
        LDA pixelYPositionLoPtrArray,X
        STA yPosLoPtr
        LDA pixelYPositionHiPtrArray,X
        STA yPosHiPtr
\end{lstlisting}

Imagine our \icode{patternIndex} is \icode{1}. This will give us the following values: 

\begin{figure}[H]
  {
    \setlength{\tabcolsep}{3.0pt}
    \setlength\cmidrulewidth{\heavyrulewidth} % Make cmidrule = 
    \begin{adjustbox}{width=7cm,center}
      \begin{tabular}{cccc}
        \toprule
        \icode{xPosHiPtr} &
        \icode{xPosLoPtr} &
        Address &
        Name \\
        \midrule
        \icode{\$0E} & \icode{\$93} & \icode{\$0E93}  & \icode{theTwistXPosArray}\\ 
        \bottomrule
      \end{tabular}
    \end{adjustbox}
  }
\end{figure}
\vspace*{-\baselineskip}

Now, using this as our basis we can read each byte from this address onwards (from each array) and
use it to calculate the X and Y position to paint a pixel at. Recall that the way we read the 
\icode{theTwistXPosArray} and \icode{theTwistYPosArray} data structures is to treat each element
as an offset from the cursor's current X/Y position. So that means taking \icode{previousCursorXPosition}
and adding the value we read from the array  to it. The way we read in a byte from the array using \icode{xPosLoPtr}
and add it to get the new position is as follows:

\begin{lstlisting}[basicstyle=\ttfamily\scriptsize]
        LDA previousCursorXPosition
        CLC 
        ADC (xPosLoPtr),Y
        STA pixelXPosition
\end{lstlisting}

\icode{(xPosLoPtr)} in the above does something very useful: it points to the address you get from combining
\icode{xPosLoPtr} \icode{(\$93)} and \icode{xPosHiPtr} \icode{(\$0E)}. It can do this because \icode{xPosLoPtr}
and \icode{xPosHiPtr} are adjacent to each other in memory.

With \icode{Y} as an offset  (Y is the counter in our loop,
incremented at each iteration) \icode{ADC} is able to point to the next byte in the \icode{theTwistXPosArray} and add it to
the accumulator, so that we can store our result in \icode{pixelXPosition}.
 
\clearpage
\begin{figure}[H]
    \centering
    \begin{adjustbox}{width=12cm,center}
      \includegraphics[width=12cm]{src/patterns/pattern2-45.png}%
    \end{adjustbox}
    \begin{adjustbox}{width=12cm,margin=0cm -4cm}
      \includegraphics[width=12cm]{src/patterns/pattern2-225.png}%
    \end{adjustbox}
\caption{'La Llamita'.}
\end{figure}
\clearpage

\rhead[]{La Llamita}
\begin{lstlisting}
laLlamitaXPosArray  .BYTE $00,$FF,$00,$55                    ;  0       
                    .BYTE $00,$00,$55                        ; 06      
                    .BYTE $01,$02,$03,$00,$01,$02,$03,$55    ;  0      
                    .BYTE $04,$05,$06,$04,$00,$01,$02,$55    ;  1    3 
                    .BYTE $04,$00,$04,$00,$04,$55            ;  12223 3
                    .BYTE $FF,$03,$55                        ;  22223  
                    .BYTE $00,$55                            ;  333 4  
laLlamitaYPosArray  .BYTE $FF,$00,$01,$55                    ;  4   4  
                    .BYTE $02,$03,$55                        ; 54  54  
                    .BYTE $03,$03,$03,$04,$04,$04,$04,$55
                    .BYTE $03,$02,$03,$04,$05,$05,$05,$55
                    .BYTE $05,$06,$06,$07,$07,$55
                    .BYTE $07,$07,$55
                    .BYTE $00,$55

\end{lstlisting}
\subfile{patterns/tables/pattern2.tex}

\clearpage
\textbf{Lines 1189-1231. \icode{\textbf{CheckKeyboardInput}}} 
\begin{lstlisting}[basicstyle=\ttfamily\scriptsize]
;-------------------------------------------------------
; CheckKeyboardInput
;-------------------------------------------------------
CheckKeyboardInput   
        ...
CheckForKeyStroke   
        LDA lastKeyPressed
        CMP #$40
        BNE ProcessKeyStroke

        ; No key was pressed. Return early.
        LDA #$00
        STA timerBetweenKeyStrokes
        JSR DisplayDemoModeMessage
ReturnFromKeyboardCheck   
        RTS 

        ; A key was pressed. Figure out which one.
ProcessKeyStroke   
        LDY initialTimeBetweenKeyStrokes
        STY timerBetweenKeyStrokes
        LDY shiftKey
        STY shiftPressed

        CMP #KEY_SPACE ; Space pressed?
        BNE MaybeSPressed

SpacePressed
        ; Space pressed. Selects a new pattern element. "There are
        ; eight permanent ones, and eight you can define for yourself
        ; (more on this later!). The latter eight are all set up when
        ; you load, so you can always choose from 16 shapes."
        INC currentPatternElement
        LDA currentPatternElement
        AND #$0F
        STA currentPatternElement
        AND #$08
        BEQ UpdateCurrentPattern
        ; The first 8 patterns are standard, the rest are custom.
        JMP GetCustomPatternElement
UpdateCurrentPattern   
        JSR ClearLastLineOfScreen
        LDA currentPatternElement
        ASL 
        ASL 
        ASL 
        ASL 
        TAY 

        LDX #$00
WritePatternDescription   
        LDA txtPresetPatternNames,Y
        STA lastLineBufferPtr,X
        INY 
        INX 
        CPX #$10
        BNE WritePatternDescription
        JMP WriteLastLineBufferToScreen
        ; Returns
\end{lstlisting}
\clearpage

\rhead[]{\icode{CheckKeyboardInput}}
\textbf{Lines 1189-1231. \icode{\textbf{SpacePressed}}:} This is the routine that detects when the player has selected a new
pattern by pressing the 'Space' key. It is part of the much larger routine \icode{CheckKeyboardInput} which periodically checks
for keyboard input by polling the byte at address \icode{\$00C5} (which we label \icode{lastKeyPressed}). This address always
contains the value of the most recently pressed key on the keyboard.

Otherwise we increment \icode{currentPatternElement}, which is the variable we use for storing the currently selected pattern.
Each press of the 'Space' key increments the value of \icode{currentPatternElement} until it reaches 15. Once that happens, we 
clamp it back to zero by using an \icode{AND \#\$0F}.

\textbf{Lines 1214-1219. \icode{\textbf{UpdateCurrentPattern}}:} What is the series of \icode{ASL} instructions doing? A tricky
piece of business of course. We have loaded \icode{currentPatternElement} to the \icode{A} register (\icode{LDA currentPatternElement}) and it has a value between 0 and 8.
\icode{ASL} performs a leftward bit-shift on the \icode{A} register. A single left-shift has an interesting property - it doubles the
value of the byte. A second left-shift will double it again, and so on. So our four {ASL} instructions have the effect of turning 1 into
16, 2 into 32, 3 into 48, and 4 into 64. This has the very useful effect of giving us an index into the description of each symmetry!

\begin{lstlisting}
txtPresetPatternNames
        .TEXT     'STAR ONE        '
        .TEXT     'THE TWIST       '
        .TEXT     'LA LLAMITA      '
        .TEXT     'STAR TWO        '
        .TEXT     'DELTOIDS        '
        .TEXT     'DIFFUSED        '
        .TEXT     'MULTICROSS      '
        .TEXT     'PULSAR          '
\end{lstlisting}

So when we take our resulting value and load it into the \icode{Y} register all our \icode{Write\-PatternDescription} needs to do is start at the 
index given by \icode{Y} and write out the next 16 bytes to the screen, displaying the selected symmetry briefly to the player.
\clearpage
\begin{figure}[H]
    \centering
    \begin{adjustbox}{width=12cm,center}
      \includegraphics[width=12cm]{src/patterns/pattern3-45.png}%
    \end{adjustbox}
    \begin{adjustbox}{width=12cm,margin=0cm -4cm}
      \includegraphics[width=12cm]{src/patterns/pattern3-225.png}%
    \end{adjustbox}
\caption{'Star Two'.}
\end{figure}
\clearpage

\rhead[]{Star Two}
\begin{lstlisting}
starTwoXPosArray  .BYTE $FF,$55                  ;    1  
                  .BYTE $00,$55                  ;   0  2
                  .BYTE $02,$55                  ;    6  
                  .BYTE $01,$55                  ; 4     
                  .BYTE $FD,$55                  ;     3 
                  .BYTE $FE,$55                  ;  5    
                  .BYTE $00,$55
starTwoYPosArray  .BYTE $FF,$55
                  .BYTE $FE,$55
                  .BYTE $FF,$55
                  .BYTE $02,$55
                  .BYTE $01,$55
                  .BYTE $FC,$55
                  .BYTE $00,$55
\end{lstlisting}
\subfile{patterns/tables/pattern3.tex}
\clearpage

\rhead[]{Deltoid}
\begin{figure}[H]
    \centering
    \begin{adjustbox}{width=12cm,center}
      \includegraphics[width=12cm]{src/patterns/pattern4-45.png}%
    \end{adjustbox}
    \begin{adjustbox}{width=12cm,margin=0cm -4cm}
      \includegraphics[width=12cm]{src/patterns/pattern4-225.png}%
    \end{adjustbox}
\caption{'Deltoid'.}
\end{figure}
\clearpage

\begin{lstlisting}
deltoidXPosArray  .BYTE $00,$01,$FF,$55           ;       5      
                  .BYTE $00,$55                   ;              
                  .BYTE $00,$01,$02,$FE,$FF,$55   ;       4      
                  .BYTE $00,$03,$FD,$55           ;       3      
                  .BYTE $00,$04,$FC,$55           ;       2      
                  .BYTE $00,$06,$FA,$55           ;      202     
                  .BYTE $00,$55                   ;     20602    
deltoidYPosArray  .BYTE $FF,$00,$00,$55           ;    3     3   
                  .BYTE $00,$55                   ;   4       4  
                  .BYTE $FE,$FF,$00,$00,$FF,$55   ;              
                  .BYTE $FD,$01,$01,$55           ; 5           5
                  .BYTE $FC,$02,$02,$55
                  .BYTE $FA,$04,$04,$55
                  .BYTE $00,$55
\end{lstlisting}
\subfile{patterns/tables/pattern4.tex}

\begin{figure}[H]
    \centering
    \begin{adjustbox}{width=12cm,center}
      \includegraphics[width=12cm]{src/patterns/pattern5-45.png}%
    \end{adjustbox}
    \begin{adjustbox}{width=12cm,margin=0cm -4cm}
      \includegraphics[width=12cm]{src/patterns/pattern5-225.png}%
    \end{adjustbox}
\caption{'Diffused'.}
\end{figure}
\clearpage

\rhead[]{Diffused}
\begin{lstlisting}
diffusedXPosArray .BYTE $FF,$01,$55                  ; 5            
                  .BYTE $FE,$02,$55                  ;            4 
                  .BYTE $FD,$03,$55                  ;   3          
                  .BYTE $FC,$04,$FC,$FC,$04,$04,$55  ;          2   
                  .BYTE $FB,$05,$55                  ; 5   1       5
                  .BYTE $FA,$06,$FA,$FA,$06,$06,$55  ;   3    0  3  
                  .BYTE $00,$55                      ;       6      
diffusedYPosArray .BYTE $01,$FF,$55                  ;   3  0    3  
                  .BYTE $FE,$02,$55                  ; 5       1   5
                  .BYTE $03,$FD,$55                  ;    2         
                  .BYTE $FC,$04,$FF,$01,$FF,$01,$55  ;           3  
                  .BYTE $05,$FB,$55                  ;  4           
                  .BYTE $FA,$06,$FE,$02,$FE,$02,$55  ;             5
                  .BYTE $00,$55
\end{lstlisting}
\subfile{patterns/tables/pattern5.tex}

\begin{figure}[H]
    \centering
    \begin{adjustbox}{width=12cm,center}
      \includegraphics[width=12cm]{src/patterns/pattern6-45.png}%
    \end{adjustbox}
    \begin{adjustbox}{width=12cm,margin=0cm -4cm}
      \includegraphics[width=12cm]{src/patterns/pattern6-225.png}%
    \end{adjustbox}
\caption{'Multi-Cross'.}
\end{figure}
\clearpage

\rhead[]{Multi-Cross}
\begin{lstlisting}
multicrossXPosArray 
        .BYTE $01,$01,$FF,$FF,$55                    ;
        .BYTE $02,$02,$FE,$FE,$55                    ;   5     5  
        .BYTE $01,$03,$03,$01,$FF,$FD,$FD,$FF,$55    ;  4       4 
        .BYTE $03,$03,$FD,$FD,$55                    ; 5 3 2 2 3 5
        .BYTE $04,$04,$FC,$FC,$55                    ;    1   1   
        .BYTE $03,$05,$05,$03,$FD,$FB,$FB,$FD,$55    ;   2 0 0 2  
        .BYTE $00,$55                                ;      6     
multicrossYPosArray                                  ;   2 0 0 2  
        .BYTE $FF,$01,$01,$FF,$55                    ;    1   1   
        .BYTE $FE,$02,$02,$FE,$55                    ; 5 3 2 2 3 5
        .BYTE $FD,$FF,$01,$03,$03,$01,$FF,$FD,$55    ;  4       4 
        .BYTE $FD,$03,$03,$FD,$55                    ;   5     5  
        .BYTE $FC,$04,$04,$FC,$55                    
        .BYTE $FB,$FD,$03,$05,$05,$03,$FD,$FB,$55    
        .BYTE $00,$55
\end{lstlisting}
\subfile{patterns/tables/pattern6.tex}
\clearpage
\begin{figure}[H]
    \centering
    \begin{adjustbox}{width=7cm,center}
      \frame{\includegraphics[width=7cm]{src/patterns/multicross_sketch.jpg}}%
    \end{adjustbox}
\caption{
  A sketch from Minter's development notes of the multi-cross pattern. Compared with the final form on the previous page
  a slight difference in the paint order is noticeable: the centre pixel is painted first in the sketch but last in Psychedelia.
  This is because the sketch is for the version of this pattern that was used in Colourspace, not Psychedelia. It's interesting
  that the reuse of the pattern was not a copy/paste exercise - instead it was redesigned from a blank page.
  }
\end{figure}
\begin{figure}[H]
    \centering
    \begin{adjustbox}{width=7cm,center}
      \frame{\includegraphics[width=7cm]{src/patterns/pulsar_sketch.jpg}}%
    \end{adjustbox}
\caption{
  From the same page in the notebook, the Colourspace 'version' of the 'Pulsar' pattern was again redrawn from scratch. Like the
  Multi-Cross the centre pixel is the first one to be drawn in Colourspace rather than the last as here in Psychedelia.
  }
\end{figure}
\begin{figure}[H]
    \centering
    \begin{adjustbox}{width=8cm,center}
      \frame{\includegraphics[width=8cm]{src/patterns/llamita_sketch.jpg}}%
    \end{adjustbox}
\caption{
  Again from the same page in the notebook, the 'Colourspace' version of La Llamita sketched out. The order of pixels is substantially
  different from that used in Psychedelia, below.
  }
\end{figure}
\vspace{1cm}
\begin{lstlisting}[caption=The paint order of pixels for 'La Llamita' in Psychedelia.]
laLlamitaXPosArray  .BYTE $00,$FF,$00,$55                    ;  0       
                    .BYTE $00,$00,$55                        ; 06      
                    .BYTE $01,$02,$03,$00,$01,$02,$03,$55    ;  0      
                    .BYTE $04,$05,$06,$04,$00,$01,$02,$55    ;  1    3 
                    .BYTE $04,$00,$04,$00,$04,$55            ;  12223 3
                    .BYTE $FF,$03,$55                        ;  22223  
                    .BYTE $00,$55                            ;  333 4  
laLlamitaYPosArray  .BYTE $FF,$00,$01,$55                    ;  4   4  
                    .BYTE $02,$03,$55                        ; 54  54  
                    .BYTE $03,$03,$03,$04,$04,$04,$04,$55
                    .BYTE $03,$02,$03,$04,$05,$05,$05,$55
                    .BYTE $05,$06,$06,$07,$07,$55
                    .BYTE $07,$07,$55
                    .BYTE $00,$55
\end{lstlisting}
\clearpage

\begin{figure}[H]
    \centering
    \begin{adjustbox}{width=12cm,center}
      \includegraphics[width=12cm]{src/patterns/pattern7-45.png}%
    \end{adjustbox}
    \begin{adjustbox}{width=12cm,margin=0cm -4cm}
      \includegraphics[width=12cm]{src/patterns/pattern7-225.png}%
    \end{adjustbox}
\caption{'Pulsar'.}
\end{figure}
\clearpage

\rhead[]{Pulsar}
\begin{lstlisting}
pulsarXPosArray .BYTE $00,$01,$00,$FF,$55       ;
                .BYTE $00,$02,$00,$FE,$55       ;       5      
                .BYTE $00,$03,$00,$FD,$55       ;       4      
                .BYTE $00,$04,$00,$FC,$55       ;       3      
                .BYTE $00,$05,$00,$FB,$55       ;       2      
                .BYTE $00,$06,$00,$FA,$55       ;       1      
                .BYTE $00,$55                   ;       0      
pulsarYPosArray .BYTE $FF,$00,$01,$00,$55       ; 5432106012345
                .BYTE $FE,$00,$02,$00,$55       ;       0      
                .BYTE $FD,$00,$03,$00,$55       ;       1      
                .BYTE $FC,$00,$04,$00,$55       ;       2      
                .BYTE $FB,$00,$05,$00,$55       ;       3      
                .BYTE $FA,$00,$06,$00,$55       ;       4      
                .BYTE $00,$55                   ;       5      
\end{lstlisting}
\subfile{patterns/tables/pattern7.tex}

\begin{figure}[H]
    \centering
    \begin{adjustbox}{width=12cm,center}
      \includegraphics[width=12cm]{src/patterns/pattern8-45.png}%
    \end{adjustbox}
    \begin{adjustbox}{width=12cm,margin=0cm -4cm}
      \includegraphics[width=12cm]{src/patterns/pattern8-225.png}%
    \end{adjustbox}
\caption{'Custom Pattern 0'.}
\end{figure}
\clearpage

\rhead[]{Custom Pattern 0}
\begin{lstlisting}
; customPattern0XPosArray                         ;
        .BYTE $00,$00,$00,$FF,$FE,$FD,$01,$02,$55 ;    33033   
        .BYTE $00,$03,$55                         ;  35  0  54 
        .BYTE $00,$00,$00,$00,$00,$55             ; 5    6    5
        .BYTE $00,$FF,$FE,$FC,$FB,$FC,$01,$02,$55 ; 3   020   4
        .BYTE $00,$04,$05,$04,$FF,$01,$55         ;    0 2 0   
        .BYTE $00,$FD,$FB,$03,$05,$02,$FE,$55     ;  30  2  14 
        .BYTE $00,$55                             ;    54245   
; customPattern0YPosArray
        .BYTE $00,$FF,$FE,$01,$02,$03,$01,$02,$55
        .BYTE $00,$03,$55
        .BYTE $00,$01,$02,$03,$04,$55
        .BYTE $00,$FE,$FE,$FF,$01,$03,$FE,$FE,$55
        .BYTE $00,$FF,$01,$03,$04,$04,$55
        .BYTE $00,$FF,$00,$FF,$00,$04,$04,$55
        .BYTE $00,$55

\end{lstlisting}
\subfile{patterns/tables/pattern8.tex}

\begin{figure}[H]
    \centering
    \begin{adjustbox}{width=12cm,center}
      \includegraphics[width=12cm]{src/patterns/pattern9-45.png}%
    \end{adjustbox}
    \begin{adjustbox}{width=12cm,margin=0cm -4cm}
      \includegraphics[width=12cm]{src/patterns/pattern9-225.png}%
    \end{adjustbox}
\caption{'Custom Pattern 1'.}
\end{figure}
\clearpage

\rhead[]{Custom Pattern 1}
\begin{lstlisting}
; customPattern1XPosArray                   ;       3      
        .BYTE $00,$00,$FF,$01,$55           ;    4  5  4   
        .BYTE $00,$FE,$02,$55               ;       6      
        .BYTE $00,$00,$FA,$06,$03,$FD,$55   ; 3     1     3
        .BYTE $00,$FD,$03,$FB,$05,$55       ;       7      
        .BYTE $00,$00,$00,$55               ;     21 12    
        .BYTE $00,$00,$FC,$04,$03,$FD,$55   ;  466     664 
        .BYTE $00,$55                       ;              
; customPattern1YPosArray                   ;    3  5  3   
        .BYTE $00,$FF,$01,$01,$55
        .BYTE $00,$01,$01,$55
        .BYTE $00,$FC,$FF,$FF,$05,$05,$55
        .BYTE $00,$FD,$FD,$02,$02,$55
        .BYTE $00,$05,$FD,$55
        .BYTE $00,$FE,$02,$02,$02,$02,$55
        .BYTE $00,$55


\end{lstlisting}
\subfile{patterns/tables/pattern9.tex}

\begin{figure}[H]
    \centering
    \begin{adjustbox}{width=12cm,center}
      \includegraphics[width=12cm]{src/patterns/pattern10-45.png}%
    \end{adjustbox}
    \begin{adjustbox}{width=12cm,margin=0cm -4cm}
      \includegraphics[width=12cm]{src/patterns/pattern10-225.png}%
    \end{adjustbox}
\caption{'Custom Pattern 2'.}
\end{figure}
\clearpage

\rhead[]{Custom Pattern 2}
\begin{lstlisting}
; customPattern2XPosArray      ;        5       
        .BYTE $00,$55          ;      8   8     
        .BYTE $00,$FD,$03,$55  ;   4         4  
        .BYTE $00,$F9,$07,$55  ;                
        .BYTE $00,$FB,$05,$55  ;                
        .BYTE $00,$00,$55      ; 3   2  9  2   3
        .BYTE $00,$00,$55      ;                
        .BYTE $00,$55          ;                
        .BYTE $FE,$02,$55      ;                
        .BYTE $00,$55          ;        6       
; customPattern2YPosArray
        .BYTE $00,$55
        .BYTE $00,$00,$00,$55
        .BYTE $00,$00,$00,$55
        .BYTE $00,$FD,$FD,$55
        .BYTE $00,$FB,$55
        .BYTE $00,$04,$55
        .BYTE $00,$55
        .BYTE $FC,$FC,$55
        .BYTE $00,$55
\end{lstlisting}
\subfile{patterns/tables/pattern10.tex}

\begin{figure}[H]
    \centering
    \begin{adjustbox}{width=12cm,center}
      \includegraphics[width=12cm]{src/patterns/pattern11-45.png}%
    \end{adjustbox}
    \begin{adjustbox}{width=12cm,margin=0cm -4cm}
      \includegraphics[width=12cm]{src/patterns/pattern11-225.png}%
    \end{adjustbox}
\caption{'Custom Pattern 3'.}
\end{figure}
\clearpage

\rhead[]{Custom Pattern 3}
\begin{lstlisting}
; customPattern3XPosArray             ;  5    
        .BYTE $00,$01,$01,$02,$55     ; 66  1 
        .BYTE $00,$00,$01,$02,$02,$55 ;  4 711
        .BYTE $00,$00,$00,$02,$55     ;   4222
        .BYTE $00,$FF,$FE,$55         ;    3 2
        .BYTE $00,$FE,$FE,$55         ;    3 3
        .BYTE $00,$FD,$FE,$55
        .BYTE $00,$55
; customPattern3YPosArray
        .BYTE $00,$FF,$00,$00,$55
        .BYTE $00,$01,$01,$01,$02,$55
        .BYTE $00,$02,$03,$03,$55
        .BYTE $00,$01,$00,$55
        .BYTE $00,$FF,$FE,$55
        .BYTE $00,$FF,$FF,$55
        .BYTE $00,$55
\end{lstlisting}
\subfile{patterns/tables/pattern11.tex}

\begin{figure}[H]
    \centering
    \begin{adjustbox}{width=12cm,center}
      \includegraphics[width=12cm]{src/patterns/pattern12-45.png}%
    \end{adjustbox}
    \begin{adjustbox}{width=12cm,margin=0cm -4cm}
      \includegraphics[width=12cm]{src/patterns/pattern12-225.png}%
    \end{adjustbox}
\caption{'Custom Pattern 4'.}
\end{figure}
\clearpage

\rhead[]{Custom Pattern 4}
\begin{lstlisting}[basicstyle=\tiny]
 ; customPattern4XPosArray      ;                 1                     
  .BYTE $00,$00,$00,$ED,$14,$55 ;                                       
  .BYTE $00,$F2,$0F,$55         ;                                       
  .BYTE $00,$00,$55             ;                                       
  .BYTE $00,$00,$55             ;                                       
  .BYTE $00,$00,$55             ;                 3                     
  .BYTE $00,$00,$FF,$01,$55     ;                                       
  .BYTE $00,$55                 ;                                       
  .BYTE $02,$55                 ;                 4                     
  .BYTE $00,$FC,$FD,$03,$04,$55 ;                                       
  .BYTE $00,$55                 ;                 5                     
                                ;                 6                     
; customPattern4YPosArray       ; 1    2      99 6106899      2    1    
  .BYTE $00,$0B,$F4,$00,$00,$55 ;                                       
  .BYTE $00,$00,$00,$55         ;                                       
  .BYTE $00,$F9,$55             ;                                       
  .BYTE $00,$FC,$55             ;                                       
  .BYTE $00,$FE,$55             ;                                       
  .BYTE $00,$FF,$00,$00,$55     ;                                       
  .BYTE $00,$55                 ;                                       
  .BYTE $00,$55                 ;                                       
  .BYTE $00,$00,$00,$00,$00,$55 ;                                       
  .BYTE $00,$55                 ;                                       
                                ;                 1                     
\end{lstlisting}
\subfile{patterns/tables/pattern12.tex}

\begin{figure}[H]
    \centering
    \begin{adjustbox}{width=12cm,center}
      \includegraphics[width=12cm]{src/patterns/pattern13-45.png}%
    \end{adjustbox}
    \begin{adjustbox}{width=12cm,margin=0cm -4cm}
      \includegraphics[width=12cm]{src/patterns/pattern13-225.png}%
    \end{adjustbox}
\caption{'Custom Pattern 5'.}
\end{figure}
\clearpage

\rhead[]{Custom Pattern 5}
\begin{lstlisting}
; customPattern5XPosArray          ;   44455566
        .BYTE $00,$00,$01,$01,$55  ;       1   
        .BYTE $00,$FF,$FF,$FE,$55  ;       1   
        .BYTE $00,$FD,$FC,$FB,$55  ;      1    
        .BYTE $00,$FD,$FE,$FF,$55  ;      7    
        .BYTE $00,$00,$01,$02,$55  ;     2     
        .BYTE $00,$03,$04,$55      ;     2     
        .BYTE $00,$55              ; 3  2      
                                   ;  33       
; customPattern5YPosArray
        .BYTE $00,$FF,$FE,$FD,$55
        .BYTE $00,$01,$02,$03,$55
        .BYTE $00,$04,$04,$03,$55
        .BYTE $00,$FC,$FC,$FC,$55
        .BYTE $00,$FC,$FC,$FC,$55
        .BYTE $00,$FC,$FC,$55
        .BYTE $00,$55
\end{lstlisting}
\subfile{patterns/tables/pattern13.tex}

\begin{figure}[H]
    \centering
    \begin{adjustbox}{width=12cm,center}
      \includegraphics[width=12cm]{src/patterns/pattern14-45.png}%
    \end{adjustbox}
    \begin{adjustbox}{width=12cm,margin=0cm -4cm}
      \includegraphics[width=12cm]{src/patterns/pattern14-225.png}%
    \end{adjustbox}
\caption{'Custom Pattern 6'.}
\end{figure}
\clearpage

\rhead[]{Custom Pattern 6}
\begin{lstlisting}
; customPattern6XPosArray             ;      3            
        .BYTE $00,$01,$02,$55         ;     3 3           
        .BYTE $00,$F6,$F6,$55         ; 2    3            
        .BYTE $00,$FB,$FA,$FB,$FC,$55 ; 2                 
        .BYTE $00,$FD,$FD,$FE,$FE,$55 ;             1     
        .BYTE $00,$05,$07,$55         ;            1      
        .BYTE $00,$F9,$F7,$FB,$55     ;           8       
        .BYTE $00,$55                 ;    6              
        .BYTE $00,$55                 ;                  5
                                      ;  6   6         5  
; customPattern6YPosArray             ;                   
        .BYTE $00,$FF,$FE,$55         ;        44         
        .BYTE $00,$FC,$FD,$55         ;        44         
        .BYTE $00,$FA,$FB,$FC,$FB,$55
        .BYTE $00,$05,$06,$06,$05,$55
        .BYTE $00,$03,$02,$55
        .BYTE $00,$01,$03,$03,$55
        .BYTE $00,$55
        .BYTE $00,$55
\end{lstlisting}
\subfile{patterns/tables/pattern14.tex}
